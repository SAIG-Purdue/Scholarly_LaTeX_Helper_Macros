%%-----------------------------------------------------------------------------
%% Helpful Custom Macros for Managing Tables
%%-----------------------------------------------------------------------------

%%-----------------------------------------------------------------------------
%% Required packages
%%-----------------------------------------------------------------------------
% Enables nicer fixed with column times
\usepackage{array}

% Enable 'sub tables' for cleaner milestones table
\usepackage{multirow}

% Needed for stripetabular environment
%\usepackage[table]{xcolor}

%%-----------------------------------------------------------------------------
%% Standard/Package Formatting Value Adjustments
%%-----------------------------------------------------------------------------
% Shrink space between captions and figure
\newcommand{\tabcapspaceshrink}{{\vspace{-6pt}}}

% Adjust the standard table name style
\renewcommand{\tablename}{\renewcommand{\baselinestretch}{1.0}\vspace{-0.0in}\footnotesize\sf\bfseries Table}

% Define a custom striped tabular environment for clear table viewing
\definecolor{lightgray}{gray}{0.85}
\newenvironment{stripetabular}{\rowcolors{2}{white}{lightgray}\tabular}{\endtabular}

%%-----------------------------------------------------------------------------
%% Custom Table Placement Macros
%%-----------------------------------------------------------------------------
% Normal Table placement
% (inputs the tabular section from the provided file)
% Usage: \addTable{filename/label}{caption}
\newcommand{\addTable}[2]{%
\begin{table}[!htb]
\caption{\footnotesize\sf #2 \label{tab:#1}}
\tabcapspaceshrink
\renewcommand{\baselinestretch}{0.95}
\begin{center}
\input{Tables/#1}
\end{center}
\renewcommand{\baselinestretch}{1}\normalsize
{\vspace{-12pt}}
\end{table}
}

% Table placement as a Horizontal page
% (inputs the tabular section from the provided file)
% Usage: \addTableH{filename/label}{caption}
\newlength{\twidth}
\newlength{\theight}

\newcommand{\addTableH}[2]{%
\setlength{\twidth}{\textwidth}
\setlength{\theight}{\textheight}

\begin{sidewaystable}
  % The following two 'custom' lengths allow for the standard text width and height to be used inside this special environment.
  \setlength{\textwidth}{\theight}
  \setlength{\textheight}{\twidth}
  \caption{\footnotesize\sf #2 \label{tab:#1}}
  \tabcapspaceshrink
  \renewcommand{\baselinestretch}{0.95}
  \begin{center}
  \hbox to \textwidth{
    \input{Tables/#1}
  }
  \end{center}
  \renewcommand{\baselinestretch}{1}\normalsize
\end{sidewaystable}
}

%%-----------------------------------------------------------------------------
%% Fixed width column types
%%-----------------------------------------------------------------------------
% Vertically centered
\newcolumntype{L}[1]{>{\raggedright\let\newline\\\arraybackslash\hspace{0pt}}m{#1}}
\newcolumntype{C}[1]{>{\centering\let\newline\\\arraybackslash\hspace{0pt}}m{#1}}
\newcolumntype{R}[1]{>{\raggedleft\let\newline\\\arraybackslash\hspace{0pt}}m{#1}}

% Vertically top focused
%\newcolumntype{L}[1]{>{\raggedright\let\newline\\\arraybackslash\hspace{0pt}}p{#1}}
%\newcolumntype{C}[1]{>{\centering\let\newline\\\arraybackslash\hspace{0pt}}p{#1}}
%\newcolumntype{R}[1]{>{\raggedleft\let\newline\\\arraybackslash\hspace{0pt}}p{#1}}
