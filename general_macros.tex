%%-----------------------------------------------------------------------------
%% Helpful package based formatting
%%-----------------------------------------------------------------------------
% Create hyperlinked/pdf bookmarked headings (commented out because not happy with special characters)
\usepackage[pdftex,bookmarks=true,bookmarksnumbered=true,colorlinks=true,linkcolor=black,anchorcolor=black,citecolor=black,filecolor=black,menucolor=black,urlcolor=black]{hyperref}

% Force figures to be put in corresponding sections
\usepackage[section]{placeins}

% Enable proper handling of URLs in text and bibliography entries
\usepackage{url}

% Enable the use of conditional operations
\usepackage{etoolbox}

% Allow stripped tablular sections and fancy colors for
\usepackage[dvipsnames,table]{xcolor}

% Allow block commenting for commenting out sections of text via the 'comment' environment
\usepackage{comment}

% Allow xspace to be used for handling most of the spacing issues around custom macros
\usepackage{xspace}

% Minimize orphans and widows
\usepackage[all]{nowidow}

%%-----------------------------------------------------------------------------
%% Custom Macros for Common file inclusions
%%-----------------------------------------------------------------------------
% Macro for loading a 'sections' tex file for the abstract
% usage: \loadAbstract{filename}
\newcommand{\loadAbstract}[1]{\input{Sections/#1}}

% Macro for loading a 'sections' tex file as a new Chapter for easier referencing
% usage: \loadChapter{filename/label}{Chapter Header Text}
\newcommand{\loadChapter}[2]{%
\newChapter{#2}{#1}%
\input{Sections/#1}%
}

% Macro for loading a 'sections' tex file as a new Section for easier referencing
% usage: \loadSection{filename/label}{Section Header Text}
\newcommand{\loadSection}[2]{%
\newSection{#2}{#1}%
\input{Sections/#1}%
}

% Macro for loading a 'sections' tex file as a new Sub-Section for easier referencing
% usage: \loadSubSection{filename/label}{Sub-Section Header Text}
\newcommand{\loadSubSection}[2]{%
\newSubSection{#2}{#1}%
\input{Sections/#1}%
}

%%-----------------------------------------------------------------------------
%% Custom Macros for Managing Sections and Referencing
%%-----------------------------------------------------------------------------
% Custom commands for easy labeling of sections for references
\newcommand{\newChapter}[2]{\chapter{#1}\label{sec:#2}}
\newcommand{\newSection}[2]{\section{#1}\label{sec:#2}}
\newcommand{\newSubSection}[2]{\subsection{#1}\label{sec:#2}}
\newcommand{\newSubSubSection}[2]{\subsubsection{#1}\label{sec:#2}}
\newcommand{\newSubSubSubSection}[2]{\paragraph{#1}\label{sec:#2}}
\newcommand{\newLiteHeader}[2]{\noindent\textbf{\textit{#1:}}\label{sec:#2}\\}

% Custom Formatted Reference Commands
\newcommand{\sectionRef}[1]{Section~\ref{sec:#1}}
\newcommand{\figureRef}[1]{Figure~\ref{fig:#1}}
\newcommand{\tableRef}[1]{Table~\ref{tab:#1}}
\newcommand{\algorithmRef}[1]{Algorithm~\ref{alg:#1}}
\newcommand{\lineRef}[1]{Line~\ref{lin:#1}}
\newcommand{\appendixRef}[1]{Appendix~\ref{app:#1}}

% Custom macro for consistently labeling lines in algorithms
\newcommand{\lineLabel}[1]{\label{lin:#1}}

% Custom Macro for formatting Footnotes
\newcommand{\newFootnote}[1]{\footnote{\renewcommand{\baselinestretch}{1.0}\scriptsize #1}}

%%-----------------------------------------------------------------------------
%% Custom Macros for Contents Lists
%%-----------------------------------------------------------------------------
\newcommand{\inventoryItem}[2]{%
\ifstrempty{#2} % If description is blank don't include the ':'
    {\item \textbf{#1}}         % Handle no description
    {\item \textbf{#1}: {#2}}   % Handle description
}

%%-----------------------------------------------------------------------------
%% Custom Macros for Working with numbers
%%-----------------------------------------------------------------------------
% Let typing "\en" be exactly the same as typing "\ensuremath". 
\let\en=\ensuremath

% Define a \ve command with two arguments, so if it called with
%     \ve an
% it will expand to
%     {\en{a_1},~\en{a_2},\ \ldots,~\en{a_{n}}}
\newcommand{\ve}[2]{\en{#1_1},~\en{#1_2},\ \ldots,~\en{#1_{#2}}}

% Macro for getting the 
\newcommand{\getint}[1]{\numexpr #1 \relax}
