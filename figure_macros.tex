%%-----------------------------------------------------------------------------
%% Helpful Custom Macros for managing Figures
%%-----------------------------------------------------------------------------

%%-----------------------------------------------------------------------------
%% Required packages
%%-----------------------------------------------------------------------------
% Enable useful conditionals
\usepackage{etoolbox}

% Allows in-line images
\usepackage{wrapfig}

% Allows large landscape full page figures
\usepackage{rotating}

% Make figure numbering in captions bold
\usepackage[labelfont=bf]{caption}

% Required for including images
\usepackage{graphicx}

%%-----------------------------------------------------------------------------
%% Standard/Package Formatting Value Adjustments
%%-----------------------------------------------------------------------------
% Shrink space between captions and figure
\newcommand{\figureCaptionSpaceShrink}{{\vspace{-6pt}}}

% Declare the path(s) where your graphic files are
\graphicspath{{./Figures/}}
% Declare the extensions so they don't have to be specified each usage
\DeclareGraphicsExtensions{.jpeg,.png,.jpg}

% Adjust the standard figure name style
\renewcommand{\figurename}{\renewcommand{\baselinestretch}{1.0}\vspace{-0.0in}\footnotesize\sf\bfseries Figure}

%%-----------------------------------------------------------------------------
%% Custom Figure Placement Macros
%%-----------------------------------------------------------------------------
% Custom macro for easily rotating figures
\newcommand*\rot{\rotatebox{90}}

%%-----------------------------------------------------------------------------
%% Floating Figure Placement Macros
%%-----------------------------------------------------------------------------

% Normal figure placement (floating placement)
% Usage: \addFigure[width]{filename/label}{caption}{optional, trim settings}
% Optionals: Optional fields must be left as {} in the call if not used
% Defaults: width = \linewidth
\newcommand{\addFigure}[4][0.96\linewidth]{%
\begin{figure}[!htb]
{\vspace{-12pt}}
\footnotesize
\centering
\ifstrempty{#4}%
	{\includegraphics[width=#1]{#2}}%
	{\includegraphics[trim=#4, clip, width=#1]{#2}}%
\figureCaptionSpaceShrink
\caption{#3 \label{fig:#2}}
{\vspace{-6pt}}
\end{figure}
}


% Full Landscape figure placement
% Usage: \insertFigureh{filename/label}{caption}{optional, trim settings}
% Optionals: Optional fields must be left as {} in the call if not used
\newcommand{\addFigureH}[3]{%
\begin{sidewaysfigure}[!htb]
\footnotesize
\centering
\ifstrempty{#3}%
	{\includegraphics[width=1.0\linewidth]{#1}}%
	{\includegraphics[trim=#3, clip, width=1.0\linewidth]{#1}}%
\figureCaptionSpaceShrink
\caption{#2 \label{fig:#1}}
\end{sidewaysfigure}
}


% Double figure placement (floating placement)
% Usage: \addFigureD{filename 1/label 1}{filename 2/label 2}{main caption}{sub caption 1}{sub caption 2}{optional, trim settings}
% Optionals: Optional fields must be left as {} in the call if not used
% Defaults: width = \linewidth
\newcommand{\addFigureD}[6]{%
\begin{figure*}[!hbt]
\footnotesize
\centering
\begin{tabular}{cc}
% Sub figure insertions
\ifstrempty{#6}%
	{\includegraphics[width=0.47\linewidth]{#1} & \includegraphics[width=0.47\linewidth]{#2} \\}%
	{\includegraphics[trim=#4, clip, width=0.47\linewidth]{#1} & \includegraphics[trim=#4, clip, width=0.47\linewidth]{#2} \\}%
% Figure sub captions
(a) #4 & (b) #5 \\
\end{tabular}
\figureCaptionSpaceShrink
\caption{#3 \label{fig:#1}\label{fig:#2}}
{\vspace{-12pt}}
\end{figure*}
}


% Quad figure placement (floating placement) 
% Due to intrinsic 9 parameter limit and the nature of a quad-plot -> full width and no input file trimming.
% Usage: \addFigureQ{filename 1}{filename 2/label 2}{filename 3/label 3}{filename 4/label 4}{main caption}{sub caption 1}{sub caption 2}{sub caption 3}{sub caption 4}
% Optionals: Optional fields must be left as {} in the call if not used
% Defaults: width = \linewidth
\newcommand{\addFigureQ}[9]{%
\begin{figure*}[!htb]
\footnotesize
\centering
\begin{tabular}{cc}
% Sub figure insertions for top 2
\includegraphics[width=0.45\linewidth]{#1} & \includegraphics[width=0.45\linewidth]{#2} \\
% Figure sub captions for top 2
(a) #6 & (b) #7 \\
% Sub figure insertions for bottom 2
\includegraphics[width=0.45\linewidth]{#3} & \includegraphics[width=0.45\linewidth]{#4} \\
% Figure sub captions for bottom 2
(c) #8 & (d) #9 \\
\end{tabular}
\figureCaptionSpaceShrink
\caption{#5 \label{fig:#1}\label{fig:#2}\label{fig:#3}\label{fig:#4}}
{\vspace{-12pt}}
\end{figure*}
}


%%-----------------------------------------------------------------------------
%% Wrapped Figure Placement Macros
%%-----------------------------------------------------------------------------

% Normal figure placement (inline w/ text wrapping)
% Usage: \insertFigure[width]{filename/label}{caption}{optional, trim settings}
% Optionals: Optional fields must be left as {} in the call if not used
% Defaults: width = 0.5 * \textwidth
\newcommand{\insertFigure}[4][0.5\textwidth]{%
\begin{wrapfigure}{R}{#1}
\footnotesize
\centering
\ifstrempty{#4}%
	{\includegraphics[width=#1]{#2}}%
	{\includegraphics[trim=#4, clip, width=#1]{#2}}%
\figureCaptionSpaceShrink
\caption{#3 \label{fig:#2}}
\end{wrapfigure}
}

% Double figure placement (floating placement)
% Usage: \addFigureD{filename 1/label 1}{filename 2/label 2}{main caption}{sub caption 1}{sub caption 2}{optional, trim settings}
% Optionals: Optional fields must be left as {} in the call if not used
% Defaults: width = \linewidth
\newcommand{\insertFigureD}[6]{%
\begin{figure}[!htb]
\footnotesize
\centering
\begin{tabular}{cc}
% Sub figure insertions
\ifstrempty{#6}%
	{\includegraphics[width=0.45\linewidth]{#1} & \includegraphics[width=0.45\linewidth]{#2} \\}%
	{\includegraphics[trim=#4, clip, width=0.45\linewidth]{#1} & \includegraphics[trim=#4, clip, width=0.45\linewidth]{#2} \\}%
% Figure sub captions
(a) #4 & (b) #5 \\
\end{tabular}
\figureCaptionSpaceShrink
\caption{#3 \label{fig:#1}\label{fig:#2}}
{\vspace{-12pt}}
\end{figure}
}

% Quad figure placement (floating placement) 
% Due to intrinsic 9 parameter limit and the nature of a quad-plot -> full width and no input file trimming.
% Usage: \addFigureQ{filename 1}{filename 2/label 2}{filename 3/label 3}{filename 4/label 4}{main caption}{sub caption 1}{sub caption 2}{sub caption 3}{sub caption 4}
% Optionals: Optional fields must be left as {} in the call if not used
% Defaults: width = \linewidth
\newcommand{\insertFigureQ}[9]{%
\begin{figure}[!htb]
\footnotesize
\centering
\begin{tabular}{cc}
% Sub figure insertions for top 2
\includegraphics[width=0.45\linewidth]{#1} & \includegraphics[width=0.45\linewidth]{#2} \\
% Figure sub captions for top 2
(a) #6 & (b) #7 \\
% Sub figure insertions for bottom 2
\includegraphics[width=0.45\linewidth]{#3} & \includegraphics[width=0.45\linewidth]{#4} \\
% Figure sub captions for bottom 2
(c) #8 & (d) #9 \\
\end{tabular}
\figureCaptionSpaceShrink
\caption{#5 \label{fig:#1}\label{fig:#2}\label{fig:#3}\label{fig:#4}}
{\vspace{-12pt}}
\end{figure}
}